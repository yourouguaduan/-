\documentclass{IEEEtran}
\title{Gaussian Process Motion Planning}
\author{Mustafa Mukadam, Xinyan Yan, and Byron Boots}
\date{}
\usepackage{amssymb}
\usepackage{amsmath}
\begin{document}
\maketitle
\textbf{\emph{Abstract}-Motion planning is a fundamental tool in robotics,
used to generate collision-free, smooth, trajectories, while satisfying task-dependent constraints. In this paper, we present a
novel approach to motion planning using Gaussian processes.
In contrast to most existing trajectory optimization algorithms,
which rely on a discrete state parameterization in practice,
we represent the continuous-time trajectory as a sample from
a Gaussian process (GP) generated by a linear time-varying
stochastic differential equation. We then provide a gradient-based optimization technique that optimizes continuous-time
trajectories with respect to a cost functional. By exploiting GP
interpolation, we develop the Gaussian Process Motion Planner
(GPMP), that finds optimal trajectories parameterized by a
small number of states. We benchmark our algorithm against
recent trajectory optimization algorithms by solving 7-DOF
robotic arm planning problems in simulation and validate our
approach on a real 7-DOF WAM arm.}
\section{INTRODUCTION \& RELATED WORK}
Motion planning is a fundamental skill for robots that
aspire to move through an environment without collision
or manipulate objects to achieve some goal. We consider
motion planning from a trajectory optimization perspective,
where we seek to find a trajectory that minimizes a given
cost function while satisfying any given constraints.

A significant amount of previous work has focused on
trajectory optimization and related problems. Khatib [1]
pioneered the idea of potential field where the goal position is
an attractive pole and the obstacles form repulsive fields. Various extensions have been proposed to address problems like
local minimum [2], and ways of modeling the free space [3].
Covariant Hamiltonian Optimization for Motion Planning
(CHOMP) [4], [5] utilizes covariant gradient descent to minimize obstacle and smoothness cost functionals, and a precomputed signed distance field for fast collision checking.
STOMP [6] is a stochastic trajectory optimization method
that can work with non-differentiable constraints by sampling
a series of noisy trajectories. An important drawback of
CHOMP and STOMP is that a large number of trajectory
states are needed to reason about small obstacles or find
feasible solutions when there are many constraints. Multigrid
CHOMP [7] attempts to reduce the computation time of
CHOMP when using a large number of states by starting
with a low-resolution trajectory and gradually increasing the
resolution at each iteration. Finally, TrajOpt [8] formulates
motion planning as sequential quadratic programming. Swept volumes are considered to ensure continuous-time safety,
enabling TrajOpt to solve more problems with fewer states.
\section{A GAUSSIAN PROCESS MODEL FOR CONTINUOUS-TIME TRAJECTORIES}
Motion planning traditionally involves finding a smooth,
collision-free trajectory from a start state to a goal state.
Similar to previous approaches [4]–[8], we treat motion planning as an optimization problem and search for a trajectory
that minimizes a predefined cost function (Section III). In
contrast to these previous approaches, which consider finely
discretized discrete time trajectories in practice, we consider
continuous-time trajectories such that the state at time \emph{t} is 

\begin{equation}
\xi(t)=\left\{\xi^d(t)\right\}^D_{d=1} \quad
\xi^d(t)=\left\{\xi^{dr}(t)\right\}^R_{r=1}
\end{equation}
where D is the dimension of the configuration space (number
of joints), and R is the number of variables in each configuration dimension. Here we choose R = 3, specifying joint
positions, velocities, and accelerations, ensuring that our
state is Markovian. Using the Markov property of the state in
the motion prior, defined in Eq. 6 below, allows us to build
an exactly sparse inverse kernel (precision matrix) [16], [22]
useful for efficient optimization and fast GP interpolation
(Section II-B)

Continuous-time trajectories are assumed to be sampled
from a vector-valued Gaussian process (GP):

\begin{equation}
\xi(t)\sim{\emph{GP}}(\mu(t),K(t,t')) \quad 
t_0<t,t'<t_{N+1}
\end{equation}
where $\mu(t)$ is a vector-valued mean function whose components are the functions $\left\{\mu^{dr}{(t)}\right\}^{D,R}_{d=1,r=1}$ and the entries $\kappa(t,t')_{{dr},{d'r'}}$ in the matrix-valued covariance function  $K(t,t')$ corresponding to the covariances between $\xi^{dr}(t)$ and $\xi^{d'r'}(t')$. Based on the definition of vector-valued Gaussian
processes [23], any collection of N states on the trajectory
has a joint Gaussian distribution,

\begin{equation}
\begin{split}
\xi\sim{N(\mu,K)},\quad \xi=\xi_{1:N}, \quad \mu=\mu_{1:N} \\
K_{ij}=K(t_i,t_j),\quad 
t_1\leq{t_i},t_j\leq{t_N}
\end{split}
\end{equation}

The start and goal states, $\xi_0$ and $\xi_{N+1}$, are excluded because
they are fixed. $\xi_i$ denotes the state at time $t_i$.

Reasoning about trajectories with GPs has several advantages. The Gaussian process imposes a prior distribution
on the space of trajectories. Given a fixed set of timeparameterized states, we can condition the GP on those states
to compute the posterior mean at any time of interest:
\begin{gather}
\bar{\xi}(\tau)=\mu(\tau)+K(\tau)K^{-1}(\xi-\mu)\\
K(\tau)=[K(\tau,t_1) \quad... \quad K(\tau,t_N)]
\end{gather}

\section{CONTINUOUS-TIME MOTION PLANNING WITH GAUSSIAN PROCESSES}
\section{EXPERIMENTAL RESULTS}
\section{DISCUSSION}
From the results in Table I we see that GPMP compares
favorably with recent trajectory optimization algorithms on
the benchmark. GPMP is able to solve all the 24 problems in
a reasonable amount of time, while solving for trajectories in
a state space 3 times the size of the state for all the other algorithms (with the exception of AugCHOMP). GPMP provides
a speedup over AugCHOMP by utilizing GP interpolation
during optimization, such that each iteration is faster without
any significant loss in the quality of the trajectory at the
end of optimization. This is illustrated in a comparison on
an example optimized trajectory obtained from GPMP and
AugCHOMP on the same problem (Figure 2).
GPMP and AugCHOMP are able to converge to better
solution trajectories (solve more problems) and do so faster
when compared to CHOMP. One advantage of these algorithms is that the trajectory is augmented with velocities and
accelerations. In contrast to CHOMP, where velocities and
accelerations are computed from finite differencing, velocities and accelerations in GPMP and AugCHOMP can be used
directly during optimization. This affects the calculation of
the objective cost and its gradient, resulting in better gradient
steps and convergence in fewer iterations.
Benchmark results show that TrajOpt is faster than our
approach and fails on only one of the problems. By formulating the optimization problem as sequential quadratic
programming (SQP), TrajOpt achieves faster convergence
with fewer iterations. However, GPMP offers several advantages over TrajOpt: the continuous representation allows
\section{CONCLUSIONS}
We have presented a novel approach to motion planning
using Gaussian processes for reasoning about continuous-time trajectories by optimizing a small number of states. We
considered trajectories consisting of joint positions, velocities, and accelerations sampled from a GP generated from
a LTV-SDE, and we provided a gradient-based optimization
technique for solving motion planning problems. The Gaussian process machinery enabled us to query the trajectory
at any time point of interest, which allowed us to generate
executable trajectories or reason about the cost of the entire
trajectory instead of just at the states. We benchmarked
our algorithm against various recent trajectory optimization
algorithms on a set of 7-DOF robotic arm planning problems,
and we validated our algorithms by running them on a 7-DOF
Barrett WAM arm. Our empirical results show GPMP to be
competitive or superior to competing algorithms with respect
to speed and number of problems solved.
\end{document}
